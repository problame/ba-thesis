\documentclass{article}
\usepackage[utf8]{inputenc}
\usepackage[inline]{enumitem}
\usepackage{parskip} % disable indentation for new paragraphs, increased margin-bottom instead
\usepackage[english]{babel}

\usepackage{hyperref}

\usepackage{todonotes}

\title{{\large Exposé for Bachelor Thesis}\\Stage-aware Scheduling in a Library OS}
\author{Christian Schwarz}
\date{October 2017}

\begin{document}

\maketitle

\section{Introduction}

Network servers account for a significant percentage of contemporary application software%
\begin{itemize*}[label={}, before={{:}}, itemjoin={{,}}]
    \item communication services (email, exchange, etc)
    \item reverse proxies
    \item message-queue systems
    \item and most importantly web-, application- and database-servers
\end{itemize*}
form the backbone of both public SaaS providers as well as corporate in-house infrastructure. 

The broad availability of cloud computing has popularized multi-tier application architectures:
the \textit{microservice} pattern encourages composition of small, scale-out services through RPC protocols.
An application component's interface is no longer defined in a programming language but merely consists of its published
RPC endpoints.
Applications backed by a database server are a prime exmaple for this kind of deployment:
the database -- being a typical bottleneck -- is usually the first to be installed on a dedicated machine.

The dominant CPUs in this market are \texttt{x86\_64} based Intel processors targeted at the mass market and tuned for an
extreme variety of workloads.
Hardware-supported operating system virtualization is used to isolate applications both for simplified administration
and increased security.
Linux is the dominant guest operating system offering a familiar environment and high-performance paravirtualization drivers.

% explain how applications are structured
The server applications running on top of this stack are primarily concerned with \emph{concurrent} request handling.
A common design pattern is to implement a sequential \textit{handler} routine to process each request, consisting of
logical sections such as protocol handling, request decoding, the actual business logic, response encoding, etc.
Concurrency is then achieved by handling each request in separate operating system thread -- either
by actually spawning a thread per request or by using a thread pool.
The sequential handler implementation is easy to understand and extend and
abstractions for the aforementioned threading models are available in virtually any programming language.
While more recent developments in language \& library design challenge this approach\footnote{coroutines, async IO, Node.js, Go},
many legacy applications still follow this model, most notably popular database systems such as MySQL.

Recent large-scale profiling at Google \cite{kanev2015profiling} suggests that the memory hierarchy of commodity
processors is suboptimal for contemporary applications:
the on-core cache size is too small to fit the working sets of the executed threads.
This \textit{cache thrashing} results in degraded performance and increases energy consumption, which is undesirable
regardless of the scale of deployment.

In section \ref{relatedwork} of this exposé, we examine previous work in this field and
identify aspects of contemporary computer architecture, software enginerring (application development specifically)
and realities of software deployment as contributors to the observed suboptimal resource utilization in today's datacenters.

Leaning on the idea of \emph{staged architectures} in DBMS design (SEDA,\cite{seda}) dating back to the early 2000s,
we propose an operating system based solution to exploit intra-query cache locality in server applications.
A key differentiator to exisiting approaches is that only minimal changes to existing application code are required.
This makes our proposal particularly attractive for large legacy code bases that cannot be trivially refactored to a
truly staged architecture as proposed by \citeauthor{seda}.

\section{Background \& Related Work}\label{relatedwork}

\subsection{Ware-house scale computing}
\todo{Elaborate on cache size problems...? already mentioned that in the intoduction, but should go into more detail?}
\subsection{Software Data Spreading}
\citeauthor{sodaspr} propose the \textit{software data spreading} technique that
``uses compiler-directed thread migration to aggregate cache capacity across cores and chips and improve [single thread] performance''\cite{sodaspr}.

The basic idea is that a thread's working set is distributed over all private on-core caches
and the thread is pinned to whichever core has the current sub-working set in cache.

Even if the sub-working sets are not totally disjoint, the authors still claim performance gains
on chips with built-in cache-to-cache transfer functionality.

\subsection{Cohort Scheduling}
\citeauthor{cohort} in turn propose \textit{cohort scheduling} where the on-core cache hit/miss ratio is improved by
\begin{enumerate}
\item partitioning a request handler code into \textit{stages}
\item grouping pending requests at the same stage into \textit{cohorts} and
\item scheduling all threads in a cohort consecutively on a processor.
\end{enumerate}

This exploits the fact that request handlers generally must be implemented to work independently of each other.
Otherwise, an attacker could easily deadlock the entire server.

To identify cohorts, the authors furthermore propose a programming model called \textit{staged computation}.
This should not to be confused with the term \textit{stage} introduced in section \ref{propsolution},
although we certainly took inspiration from it.

\subsection{Unikernels \& Library Operating Systems}
\todo{briefly explain unikernels, reference mirageos paper, lto, performance \& security benefits, etc}
\subsubsection{OSv}
\todo{I think we have to mention it...? How much detail?} 

\subsection{Pipeline Scheduler by Gottschlag}

In a proof-of-concept implementation, \citeauthor{gottschlag2017} combined the approach of \textit{software data spreading} and \textit{cohort scheduling}:
\todo{how does mathias' pipeline scheduler actually work?}
\todo{what are the problems of implementing it like mathias did?}

\section{Proposed Solution}\label{propsolution}

The goal of this thesis is a clean implementation of the approach by \citeauthor{gottschlag2017} in the OSv library operating system.
Subsequently, the implementation shall be evaluated by comparison to unmodified OSv and an up-to-date Linux installation.

We define the concept of \textit{stages} leaning onto the proposal from \cite{cohort}:
\begin{quote}
A \textbf{stage} describes a logical section of a program that distinguishes itself from other logical sections in the program
through a (mostly) disjoint working set.
\end{quote}

In server applications, the request handling code is the primary target for stage partitioning.
Let us consider the request handler of a simple HTTP service that converts images to thumbnails:
{
\newcommand{\stage}[1]{\texttt{#1}}
\begin{description}[parsep=0pt,labelwidth=1cm,itemindent=1cm]
    \item[\textit{\stage{stage}}] \textit{stage description}
    \item[\stage{limit}] client rate limiting / blacklist
    \item[\stage{recv}] buffer request (up to max request size) 
    \item[\stage{dec}] decode request parameters (JSON) \& image data
    \item[\stage{auth}] user authentication \& authorization against user database service
    \item[\stage{proc}] image processing / thumbnail generation
    \item[\stage{save}] store thumbnails in download directory
    \item[\stage{enc}] construct JSON response with download URLs
    \item[\stage{send}] send response to client socket
\end{description}
Obviously, \stage{recv}, \stage{auth}, \stage{save} and \stage{send} will be IO-bound
while \stage{rl}, \stage{dec} \stage{enc} and in particular \stage{proc} are likely CPU-bound.

It is crucial to observe that the majority of IO-bound code is part of OSv, not the application.
Nonetheless, this code is still part of the request handler's working set and must be taken into account when defining stages.
}

% TODO: why not just increase cache size? -> market needs, applications will grow cache footprint in turn, etc
%       handling of one thrashing thread while all others fit in


% unikernels & osv
% software data spreading techniques
% cohort scheduling
% relation to performance analysis: flame graphs?
% mathias' demo implementation


% Our solution to the aforementioned shortcomings is based on a new scheduling policy leveraging knowledge about a thread's current stage:

\subsection{Work Plan}

Based on the aforementioned concepts, our work plan for this thesis lays out as follows:

% The concrete work plan lays out as follows:
\begin{enumerate}
    \item \label{stageapi} Define a C/C++ API to partition application code into stages.
    \item \label{appdesign} Find an existing server application $A$ that
    \begin{enumerate}
        \item handles requests by spawning or re-using an operating system thread,
        \item keeps this relationship between thread and request intact until the request handler has finished executing,
        \item \label{app_osv_image} can be compiled and built into an OSv virtual machine image.
    \end{enumerate}
    \item Explore heuristics to identify stages in application code.
    \item Partition $A$'s code into stages using aforementioned API.
    \begin{enumerate}
        \item Define a number of $S_A \ge 1$ stages.
        \item Assert that a thread is always in the correct stage $S_i \in S_A$.
    \end{enumerate}
    \item \label{redo} Implement a \textit{stage-aware scheduling policy} in OSv (see section \ref{stageawaresched}).
    \item Evaluate the implemented solution (see section \ref{evaluation}).
    \item If necessary, goto \ref{redo}.
    \item Repeat the above with other suitable applications. (Optional)
    \todo{Stagify OSv?}
\end{enumerate}

Work item \ref{stageapi} has already been started by \citeauthor{gottschlag2017}.

The obvious candidate for an existing application is the MySQL database server:
databases are typical bottlenecks in web applications and usually the first to be installed on a dedicated machine\todo{buzzword: multi-tier architecture}.
MySQL implements the threading model outlined in \ref{appdesign} (see \cite{mysqlThreading}),
has already been used in \citeauthor{gottschlag2017}'s proof of concept
and been ported to OSv. \label{willusemysqlfirst}

Stack-sampling has proven as a useful heuristic for identifying stages.
The ability to instrument the memory hierarchy using performance counters should also be investigated.

\subsubsection{Stage-Aware Scheduling}\label{stageawaresched}

The major effort of this thesis will lie in the implementation of the \emph{stage-aware scheduler}, the development of a suitable \emph{policy} as well as its evaluation.

The basic idea behind our proposal is to leverage knowledge about a thread's current stage when making scheduling decisions:
\begin{quote} %TODO right env
A ready thread in stage $S_i$ should be scheduled on a processor $P$ that last ran a thread also in stage $S_i$.
\end{quote}

To maintain tractability of this problem, we assume a fixed number of $P$ symmetric processors with a uniform memory architecture.
NUMA architectures do not receive special attention in this thesis. \todo{justify this?}% TODO justify? basically, the NUMA costs on thread migration must be taken into account... what else?

At the time of writing this exposé, the implementation details have not been figured out yet.
Thus, the remainder of this section presents aspects that should be kept in mind when designing and implementing the scheduling policy:

The aforementioned \textit{stage C API} is effectively an additional entry point to the OSv scheduler.

By scheduling a stage on the same core, the pre-warmed cache and
the assumed correct partitioning of the thread's stages
will lead to a reduction in cache misses and thus better performance.
It should be apparent that this idea borrows heavily from \cite{sodaspr,cohort}.

Recall variables \textbf{$S$} (all stages), \textbf{$P$} (all processors) and \textbf{$T$} (\#threads):\\
If $|S| \le |P|$, a first draft of the scheduling policy would
assign each stage $S_i$ to a single $P_i$ and schedule threads in $S_i$ only the corresponding $P_i$.

By spreading a thread's distinct working sets over $P$ processors,
the effectively available cache size is increased up to a factor of $|P|$, depending on
\begin{enumerate*}
    \item how disjoint the stage's working sets actually are and
    \item that the size of each stage's working set is smaller than the on-core cache size ($size(S_i) \overset{!}{\le} size{on\_core\_cache}$)
\end{enumerate*}.

However, if $S > P$
\begin{itemize}
    \item \textit{victim stages} must be found that are a scheduled on the same processor,
    \item the effectiveness of the choice of victim stages could be evaluated using CPU performance counters and
    \item given this knowledge, other victim stages could be chosen.  
\end{itemize}

Apart from the mapping of $S \mapsto P$, it is an open question how well the proposed technique scales with the number of stages and each stages working set size $size(S_i)$.
We assume there exists a break-even value for $size(S_i)$ below which migration costs exceed the benefits reaped from the proposed policy.
% (Working sets will not be $100\%$ distinct, cache misses will occur, cache coherence protocol costs, etc). 

Also, the proposed 1:1 mapping $S \mapsto P$ leads to suboptimal resouce utilization if any stage $S_c$ is a point of congestion:
threads will compete for the processor $P_c$ and under-utilize all other processors $P_d$ where $d \ne c$.

An obvious solution is to assign $S_c$ to several $P_{c,i}$, introducing another scheduling dimension.
However, this still only works if the average stage execution times are of the same order of magnitude.

A last aspect are threads not spawned by the application but by OSv (traditionally called \emph{kernel-only threads}),
e.g. file system worker threads.\todo{make it clear we don't mean the upper half of FS code executed by a syscall}
Ideally, these should also be adapted to the staging API although we figure time will not allow for this to happen.

We can conclude that the simple stage-aware scheduling policy outlined above is rather specific and
is likely suboptimal for a multi-purpose / multi-user operating system.

With OSv though, we are targeting a library OS:
the entire system is running a single server app and all resources dedicated to the virtual machine are available for that purpose.

\subsection{Evaluation}\label{evaluation}

\subsubsection{Environment}\label{environment}
Before discussing the details of how we evaluate our implementation,
we need to investigate how the usage of a library OS constraints our benchmarking environment:

\begin{itemize}
    \item The proposed scheduling policy explicitly targets server applications.
    \item A benchmark for a server application does not behave like a server application.
    \item Thus, executing the benchmark \emph{inside} the OSv VM means that we
    \begin{enumerate}
        \item would need to adapt the benchmark to the stage API or
        \item would need a special-case in the scheduler for the benchmark.
    \end{enumerate}
    \item Either way, distorted benchmark results are to be expected.
\end{itemize}

The virtual machines used for the benchmarks are configures as follows:
\begin{enumerate}
    \item Unmodified OSv + unmodified application
    \item Stage-aware OSv + modified application
    \item \label{linuxvmsetup} Linux-based VM + unmodified application
\end{enumerate}
The hypervisor is planned to be Linux KVM using \texttt{virtio-*} paravirtualization whenever possible.

We recognize that running the Linux setup in a VM may not be a fair comparison:
OSv \emph{requires} execution in a virtualized environment due to lack of drivers
but Linux could also use the host's hardware directly (and use cpu\_affinity settings for comparable benchmarks).\\
If performance gains in OSv do not significantly exceed the expected virtualization overhead, this aspect should be reconsidered.

Lastly, we want to eliminate differences in the Disk IO subsystems from the benchmarks:
in both the OSv and Linux VMs, the file system storing database data must be RAM-based.
\todo{Alternatively: Since OSv only has ZFS as alternative, run ZFS on LInux inside linux vm?}

\subsubsection{Benchmarks}

Given the environment outlined above, we propose to execute \textbf{black-box benchmarks} against the virtual machines.
The primary success metrics are
\begin{itemize}
    \item request latency
    \item total request duration
    \item standard deviation of the above.
\end{itemize}

These metrics are suitable to measure the practical advantage of our implementation but
do not validate our assumption that performance gains result from improved cache hit/miss ratio.

We plan to use performance counters to measure cache hit/miss ration during the benchmark.
However, further investigation is required to decide whether this should be implemented on the hypervisor or in the guest OSes themselves
and how this data is merged with the benchmark results.

\subsubsection{MySQL Benchmarks}

As outlined in \ref{willusemysqlfirst}, the first application to be adapted to stage-aware scheduling is the MySQL database server.

The \textit{OLTPBenchmark} project (see (\cite{oltpbench})) combines several RDBM benchmarks.
\todo{which exactly? should this be mentioned here?}
Unfortunately, most are designed to be executed on the same machine as the database server, which does not work with OSv (see \ref{environment}).
Thus, we expect the need to customize the benchmark code with regards to this issue.

\subsubsection{Synthetic Application}

In addition applications such as MySQL, a less complex, synthetic application seems appropriate.

\subsubsection{Other Applications}

It should be noted that if time allows for it, other applications with readily usable benchmarks could be ported.

\clearpage
\bibliographystyle{alpha}
\bibliography{bib.bib}

\end{document}
