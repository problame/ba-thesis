\documentclass[12pt,a4paper]{book}
\usepackage[utf8]{inputenc}
\usepackage[inline]{enumitem}
\usepackage{parskip} % disable indentation for new paragraphs, increased margin-bottom instead
\usepackage[american,ngerman]{babel}

\usepackage{kit_style/kitthesiscover}

\usepackage[style=alphabetic]{biblatex}
\addbibresource{bib.bib}

\usepackage[%dvipdfm,
   pdfauthor={Christian Schwarz},
   pdftitle={Stage-aware Scheduling in a Library OS},
   pdfsubject={Bachelor Thesis},
   pdfkeywords={Operating Systems, Library OS, Scheduler, Cache-Affinity}
]{hyperref}

\usepackage{todonotes}
\usepackage{blindtext}

\begin{document}
\frontmatter
\unitlength1cm
\selectlanguage{american}

\title{Stage-aware Scheduling in a Library~OS}
\author{cnad. inform. Christian Schwarz}
\thesistype{Bachelor Thesis}
\primaryreviewer{Prof.\ Dr.\ Frank Bellosa}
\advisor{M.\ Sc.\ Mathias Gottschlag}{}
\thesisbegindate{XX.\ December 2017}
\thesisenddate{XX.\ March 2018}
\maketitle

\begin{otherlanguage}{ngerman}
\thispagestyle{empty}
\vspace*{30\baselineskip}
\hbox to \textwidth{\hrulefill}
\par
\noindent Ich versichere wahrheitsgem"a"s, die Arbeit selbstst"andig verfasst, alle benutzten Hilfsmittel vollst"andig und genau angegeben und alles kenntlich gemacht zu haben, was aus Arbeiten anderer unver"andert oder mit Ab"anderungen entnommen wurde sowie die Satzung des KIT zur Sicherung guter wissenschaftlicher Praxis in der jeweils g"ultigen Fassung beachtet zu haben.\\

\noindent Karlsruhe, den \today
\end{otherlanguage}


\chapter{Abstract}
\chapter{Acknowledgments}

\mainmatter
\cleardoublepage
\phantomsection
\addcontentsline{toc}{chapter}{Contents}
\tableofcontents

\chapter{Introduction}
Network servers account for a significant percentage of contemporary application software and play a central role in today's public and corporate infrastructure.
This class of application software is primarily concerned with handling of concurrent requests.
A common design pattern is to implement a \emph{sequential} handler routine to process each request, consisting of logical stages such as protocol handling, decoding/encoding, actual business logic, etc.
Concurrency is achieved by executing the handler code per connection in separate operating system thread -- either by spawning a thread per request or by using a thread pool.
The sequential handler implementation is simple to follow for programmers and abstractions for the aforementioned threading models are available in virtually any programming language and operating system.\todo{ref}

The dominant CPUs in the server market are \texttt{x86\_64} based Intel processors targeted at the mass market.
The memory hierarchy of processors since the Haswell microarchitecture features separate 32KiB L1-d and L1-i caches and a unified 256KiB L2 cache per core, as well as a shared inclusive L3 cache that is significantly larger.\todo{ref}
%Hardware-supported operating system virtualization using KVM or Xen is commonly used to isolate applications both for simplified administration and increased security.
%Linux is the dominant guest operating system offering a familiar environment and high-performance paravirtualization drivers.
Recent large-scale profiling at Google suggests that the memory hierarchy of these commodity processors is suboptimal for the server applications they run:
the request-handlers' instruction working-set does not fit into the on-core caches, ultimately leading to pipeline stalls while the processor waits for slower L3 accesses.\cite{kanev2015profiling}

Previous work in this field includes techniques such as \emph{cohort scheduling}:
application code is divided into execution stages and batches of threads (\emph{cohorts}) in the same stage are scheduled in series to reap the benefits of warm instruction caches.
\emph{Staged event-driven architecture} (SEDA) is an alternative approach: request processing state is encapsulated into an object which is passed through a pipeline of stages.
Each stage performs its work on the object and enqueues it into the next stage's queue, using a usually fixed number of worker threads.
If the workers' instruction working sets fit into the on-core caches, instruction-dependent pipeline stalls are greatly reduced.
SEDA has been implemented in a research DBMS under them term \emph{Steps}, yielding an overall speedup of $1.4$ in the TPC-C benchmark.\cite{cohort,seda,harizopoulos2004steps,harizopoulos2005staged}

Despite the promising benchmark results of SEDA and Steps, it has not found adoption in popular database management systems such as MySQL, which uses thread-basaed concurrency (one handler per request) with a pre-spawned pool of handler threads.\cite{mysqlThreading}.
We can only attribute this to the amount of refactoring work it would take to adopt SEDA in MySQL.
Under the assumption that the amount of refactoring work required for SEDA impedes adoption in many legacy applications, members of the \todo{phrasing} KIT OS group experimented with an approach that requires minimal customization of existing applications:
the application develpoer defines stages and inserts one-line stage switching calls into the request handling code.
By dedicating CPU cores to stages and migrating threads between the cores on stage switch, a prototype implementation in Linux reduces instruction cache misses by X\% \todo{data}.
However, experiments show that the implementation is not work-conserving on multi-processor systems.\todo{ref}

This thesis shows that proper OS abstractions enable adoption of staged computation in legacy applications with minimal customization effort:
\begin{itemize}
    \item We analyze the design and implementation of the Linux proof-of-concept and point out why it is not work-conserving.
    \item We present OS abstractions and a user-space C++ API for applications to manually define stages and stage switching points in the request-handling code path.
    \item We present the design and implementation of a work-conserving scheduling policy that allocates CPUs to stages and migrates threads as necessary to preserve warm instruction-caches.
    \item We show that appropriate stage partitioning and switching points lead to a reduced cache miss rate of X\% in synthetic benchmarks and throughput gains of X\% in a stage-aware version of MySQL under the TPC-C benchmark.\todo{measurement}
\end{itemize}

The remainder of this thesis is structured as follows:
In Chapter~\ref{ch:relwork}, we examine preceding work in the area of staged computation.
Chapter~\ref{ch:analysis} dissects the KIT OS group's proof-of-concept implementation and points out why the design is not work conserving.
Subsequently, Chapter~\ref{ch:di} presents the design and implementation of our solution in the OSv library operating system.
Finally, we evaluate our implementation in Chapter~\ref{ch:eval}, using both synthethic microbenchmarks and the TPC-C benchmark against a stage-aware version of MySQL to show the intended cache behavior and performance improvements.

\chapter{Related Work}\label{ch:relwork}
The idea of staged computation has surfaced several times since the early 2000s, motivated by the still growing demand for high-performant network servers.
This chapter starts with a summary of this preceding work, describing the design and implementation challenges encountered by the respective authors.
We present implementations of staged computation in research databases that show performance benefits due to reduced i-cache misses and branch prediction.
We finish our investiation of staged compution by investigating the underwhelming\todo{word} adoption in production systems by analyzing development discussions of popular open-source projects implementing network servers.

Shifting forward several years of research, we present profiling results showing that large instruction-cache footprint is still a major cause of performance bottlenecks in contemporary scale-out applications.
We summarize existing techniques to spread the \emph{data} working set of a process over multiple CPU caches and finally present an existing prototype at the KIT OS Group that applies the spreading idea to the \emph{instruction} working set of a thread.

\section{Cohort Scheduling}
\blindtext

\section{SEDA}
\blindtext

\section{STEPS}
\blindtext

\section{Adoption of staged computation}
% NGINX is event-driven, think it counts as such
% Apache is not event driven, maybe check MLs for analysis of nginx
% MySQL definitely not, maybe search ML posts for SEDA papers, also check Postgres
\blindtext

\section{Profiling at Google (Kanev et al)}
\blindtext

\section{Memory Hierarchy on Contemporary Server-Class CPUs}
\blindtext

\section{Software Data Spreading}
\blindtext

\section{Protoype in Linux at KIT OS Group}
%TODO this section should coin the term stage-aware scheduling, otherwise, we need to do that before the analysis --- rename section with stage-aware in title
A Linux-based proof of concept implementation at the KIT OS group combines the idea of staged execution, cohort
scheduling and software data spreading into a C API that allows for intuitive conversion of existing code bases to
staged execution:
The developer manually identifies stages and inserts library calls into application code for switching the current stage.
Each CPU core is assigned one or more stages and each time a thread switches to a new stage, it is migrated to a core
assigned to that stage.

It is obvious that a fast thread migration mechanism is required for this technique to succeed
--- otherwise, the performance benefits of always-warm caches per stage are destroyed by the migration time.
%\newcommand{\setaffinity}{\texttt{sched\_setaffinity(2)}}
The Linux built-in facility for this purpose, \texttt{sched\_setaffinity}, is impractical because it uses
expensive inter-processor-interrupts to implement this syscall, resulting in $9\mu s$ -- $14\mu s$ of migration time.~\cite{sodaspr}% TODO check on IPs 
As a consequence, thread migration was implemented in user-space:
for each user-level thread (ULT), there still exists exists a kernel-level thread (KLT),
but KLTs are pinned once to a specific core and stage.
ULTs run on a KLT of the stage they are currently in and migrate to a different KLT when switchting stages.
TODO threading model required
TODO automatic -> mathias' paper, when it's published

When a ULT makes a call to switch stages, its context is saved and enqueued into the next stage's ready queue.
The originating KLT now waits for new ULTs on its own stage's incoming migrations queue.
If it is empty, the KLT makes a blocking syscall \texttt{TODO\_dequeue\_syscall} to a kernel component to wait for incoming migrations.
The kernel component must ensure that there is always one KLT per core either doing work or actively dequeuing ULTs in order to utilize the CPU.
This is implemented by a callback from the Linux scheduler that informs the kernel component of task state changes.
For example, if the currently dequeuing KLT $K_1$ executes a ULT, and that ULT blocks on a mutex, $K_1$ switches to \texttt{TASK\_INTERRUPTIBLE}.
The kernel component must wake up another KLT $K_2$ that is currently waiting for incoming migrations of that stage on that core.
Otherwise, the core does not perform any work (for the application) until $K_1$ aquires the mutex and switches back to \texttt{TASK\_RUNNING}

TODO results with single threaded execution, show that it works, reduced cache misses and performance gains are visisble.

However, multithreaded workloads on an SMP system exhibit visible CPU underutilization.
Imagine a system with $\frac{\#cores}{stage} > 1$ and a ULT $U_1$ in stage $S$ executing on a KLT $K_1$.
As described above, $K_1$ is pinned to core $C_1$.
When $U_1$ performs a blocking syscall, for example waiting to aquire a mutex via \texttt{pthread\_mutex\_lock}, $K_1$ blocks and becomes \texttt{TASK\_INTERRUPTIBLE}.
The Linux scheduler now dispatches another task $T$ on $C_1$ to maximize CPU utilization.
(Note that $T$ is not a $K_i$ of our application. In fact, all $K_i$ are blocked in \texttt{TODO\_dequeue\_syscall}.)
When $K_1$ finally aquires the mutex and is \texttt{TASK\_RUNNING} again, it can still only be dispatched to $C_1$ due to pinning which is necessary for user-level thread migration.
However, $C_1$ will be executing $T$, not $K_1$ which is put into $C_1$'s ready queue instead.
The misery at this point is that there may exist \textbf{another} CPU $C_2$ where a KLT $K_2$ is dequeuing ULTs in the \textbf{same stage} $S$:
if $K_2$ does not have any ULTs to execute, $K_1$ should be migrated to $C_2$ immediately when it is woken up and continue execution there, benefiting from the warm on-core caches.
But the implementation only performs thread migration when a ULT calls the stage switching API.
There is no mechanism in place to save $K_1$'s state and enqueue it to $K_2$'s incoming migration queue on wake-up.
(One might assume it is possible to enqueue $U_1$ to $K_2$ since we saved its register state on kernel entry via \texttt{pthread\_mutex\_lock}:
this is not possible because there might still be kernel code that needs to run after the mutex is aquired, before returning to $U_1$.)

TODO UML sequence diagram visualizing the pathological case described above.

\chapter{Analysis}\label{ch:analysis}
Related work has demonstrated that spreading the working set of an application over multiple cores yields lower on-core cache miss rates.
The proof-of-concept implementation at the KIT OS group furthermore shows that large-scale refactoring of existing code bases is not necessarily required to spread the working set:
given an application with a suitable threading model, patches of a few lines are sufficient to reduce the cache-miss rate by TODO.

Nonetheless, we identify several fundamental problems in the approach taken by the proof-of-concept implementation:
the requirement for fast thread migration drove the design toward a user-space solution which decouples the threads known by the application (ULTs) from the threads known by the kernel (KLTs).
However, the kernel scheduler still only handles KLTs and assumes a 1:1 threading model, which leads to an ambiguous role of KLTs in the proof-of-concept:
when switching between stages, the user space thread migration code views KLTs as the CPUs they are pinned to.
But when a ULT running on a KLT interacts with the Linux kernel, the kernel sees a normal \texttt{task\_struct} and continues to assume the 1:1 threading model where tasks can just block.
The proof-of-concept works around this schizophreny by introducing a callback from the scheduler to react to blocking KLTs, but fails to handle asynchronous events like wake-ups.
The latter leads to the observed pathological behavior on multi-core systems.

We conclude that the proof-of-concept does not model the situation correctly: the association of stages and CPU cores is piggybacked onto the KLTs using \texttt{sched\_setaffinity}.
Instead, we prpose that stages must be modeled explicitly in the kernel and be separate entities, beneath CPUs and threads:
\begin{itemize}%
    \item The associaton of stages and CPU cores must be represented explicity.
    \item Threads must carry the information in which stage they are executing.
    \item The scheduler must honor this information by scheduling threads onto cores that are associated with their respective stage.
    \item The scheduler must trade off the potential gains of always-warm caches against existing scheduling goals such as resource utilization, fairness and response time.
\end{itemize}%
This reverts the complex situation of ULTs and KLTs to a simple 1:1 threading model, removing the special-case of blocking kernel activity.

The remainder of this thesis presents our design as well as its implementation in the OSv library operating system and its evaluation using several microbenchmarks and the TPC-C benchmark against the MySQL database management system.

\chapter{Design \& Implementation}
In this chapter, we present the design and implementation of our solution to the proposal in Chapter~\ref{ch:analysis}.
Our design is limited to server applications with high instruction cache footprint and a thread-pool or thread-per-request threading model:
related work has shown that high instruction cache miss rates are a performance bottleneck in scale-out applications
and demonstrated lower cache miss rates in the MySQL database server, which follows the thread-per-request model with a pool of pre-spawned threads.
We re-use their tested stage definitions in MySQL \todo{KIT OS group} and focus on the implementation of a work-conserving scheduling policy.
\cite{kanev2015profiling,mysqlThreading}\todo{KIT OS group}

The limited compatibility with threading models is not a major limitation of our design:
our approach aims at reaping the benefits of staged computation in legacy applications that cannot be easily refactored to a staged-computation model.
Thread-pools and per-request threads were the dominant threading model in the last TODO years and are in fact still used in new applications today.
We acknowledge that more recent runtimes with event-driven concurrency or M:N scheduling models are incompatible with our approach, unless programmers manually construct the supported threading models on top of them.\todo{proof}\todo{discussion section?}

The following enumeration provides an overview of our contributions:
\begin{enumerate}
    \item We implement our solution in the OSv library operating system, which provides Linux-ABI compatibility and runs on Linux/KVM hypervisor.~(Section~\ref{ch:di:osv})
    \item We remove the existing load balancing and fairness mechanisms from OSv's scheduling policy, replacing it with non-preemptive round-robin per CPU core.~(Section~\ref{ch:di:rmsched})
    \item We add a user-space API to define stages and switch between them.~(Section~\ref{ch:di:api})
    \item We implement a fast thread migration mechanism that does not use inter-processor interrupts.~(Section~\ref{ch:di:mig})
    \item On wake-up of a thread, we load-balance between the CPU cores of its current stage, using thread migration as necessary.~(Section~\ref{ch:di:workcons})
    \item We add an allocation policy for CPUs to stages that maximizes throughput by removing bottleneck stages.~(Section~\ref{ch:di:pol})
\end{enumerate}
Each section starts with the reasoning behind our design followed by our implementation strategy in OSv as well as the refactorings it required.
We refer to the relevant commits in the Git repository of our modified version of OSv as appropriate.

\section{The OSv Library Operating System}\label{ch:di:osv}
% what is osv, how does it work:
%       libc + linux ABI => unmodified, existign software runs (mention ports), etc
%       compact scheduler implementation => see below
% why OSv for this thesis:
%    small kernel / scheudler => implementation complexity
%    virtual machines         => Linux scheduler for fairness goals, our OSv scheduler can focus on optimal stage-aware scheduling
%
% Design Deficiencies:
%           While offloading almost all other design goals to Linux scheduler makes implementation considerably easier,
%           it is by far not optimal: like all nested dynamic systems, the Linux scheduler and our scheduler can inadvertently work against each other
%           => for simplicitly, we limit our design to a solution where cores are dedicated to the OSv VM

\begin{itemize}
    \item Concept of a library OS: vm = app, no user-kernel-boundary => performance promise
    \item Linux ABI-compat
    \item Moderately sized C++ codebase, compact scheduler implementation
\end{itemize}
Idea: OSv allows for easier replacement of the entire scheduling policy and and thread migration mechanisms.
By the nature of a library OS, the changes are kept application-local, leaving fairness considerations up to the hypervisor scheduler (Linux / KVM in our case).

\subsection{The OSv Scheduler}
This section highlights implementation details of the upstream OSv scheduler that are required to understand the framework in which our implementation was done.
For details, we refer to the TODO paper.
\begin{itemize}
    \item runqueues + idle thread
    \item thread pinning => percpu threads
    \item thread migration mechanism
    \item thread states => state diagram
\end{itemize}

\section{Removal of OSv Scheduling Policy}\label{ch:di:rmsched}
% remove load balancing + fairness (cpu time, priority queue / sorted list? )
% stub out pthreads_sched_setaffinity_np + remove dead thread migration code (why did we do that?)
% implement simple CPU-local round-robin 
% describe outcome:
%   cpu-local runqueues, round robin, non-preemption (OKish because threads always finish, while not really fair...)
%   existing thread migration mechanism still required by percpu threads, but only on startup
%   idle threads are always runnable, dequeue incoming migrations per CPU
%   
\blindtext
\section{User-Space API}\label{ch:di:api}
% Define stages as objects in the kernel
% User-space can enqueue itself to a stage, will be migrated to one of the cores it is assigned to
% Stack structure for implementation comfort, describe idea behind stack structure & guards -> spans, like in web browser
\blindtext
\section{Fast Thread Migration}\label{ch:di:mig}
% Motivation: Stage switch requires migration to other core
% Existing migration mechanism is based on IPIs, is too slow => results, maybe also compare to Linux => adjust hopper
% Must be fast, because of opportunity cost calculation:
%   -> for single client:       cold cache and no migration vs warm cache but migration times
%   -> for T > #cpus clients: ???
% Design:
%       stage migration and hence migration to other CPU core is synchronous to program flow
%       MPSC queue per CPU, for incoming stage migrations
%       on stage switch, evaluate CPU allocation policy, pick target CPU, enqueue into stagesched_incoming
%       target CPU dequeues from stagesched_incoming in idle thread -> mention potential optimization: mwait
%       Measurement results: why not here??? specifically if we discuss mwait?
% Implementation:
%       a stage switch means scheduling out on the source CPU, being migrated to the target CPU and resuming execution there
%       per-cpu MPSC queue that contains pointers to TCBs of threads in stage migration
%       cpu idle thread dequeues from mpsc
%       additional thread states for thread in stage migration
%       critical race condition:  only after scheduling out on the source CPU must execution continue on the target CPU,
%           but thread puts itself into the MPSC queue while it is still executing
%           => augment the thread-switching routine (which is the last code the migrating thread execute on the source CPU) to atomically
%              switch the state of the thread from stagemig_run to stagemig_sto
%           => target CPU only dequeues those threads that are in stagemig_sto to its runqueue, others are still executing code on the source CPU
% => State diagram with changes made in this step
\blindtext
\section{Work Conservation}\label{ch:di:workcons}
% Motivation: with thread migration mechanism in place, we are faster than the Linux solution but not still work conserving
%             when a thread is woken up, it will be enqueued inot the runqueue of the CPU where it went to sleep
%             if that CPU is already executing a thread but another one of the same stage is idle, we do not use the available resources efficiently
%             the woken thread should be migrated to the idle CPU and immediately resume execution there
%             important: this is not stage switching, but also needs to evaluate the CPU allocation policy (forward ref!)
% Design:
%             on wakeup, evaluate the CPU allocation policy and migrate the woken thread to that CPU using the normal thread migration mechanism
% Implementation:
%             wakup is asynchronous, e.g. when from a timer 
%             we must only migrate a thread when it is stopped, otherwise it's already running and does not need to be woken up
%             but there are states in which the thread is being woken up while going to sleep
%             => cleanly encode running vs stopped in the thread state
%               => more difficult than expected
%               => extend post-switch-to mechanism with appropriate state transitions
%               => TODO see git commit
%               => state diagram with changes made in this step
%             Afterwards, just use the existing thread migration mechanism TODO check if we can go without the IPIs
\blindtext

\section{Stage-Aware Scheduling Policy}\label{ch:di:pol}
% Motivation: bottleneck stages = stages with more-than-avergage threads in the ready queues of processors assigned to these stages
%             natural observation: stage needs more CPU time / cores in our case
%             alterantive formulation (is that correct?):
%                   the time it takes from begin of stage switch to first dispatch on CPU in new stage should be equal among all stages
%                   => predictable latencies
% Design:
%            among all stages, equalize the number of threads that are currently executing / enqueued in each of them
%            why?
%                   => obvious that this eleminiates bottleneck stages
%                   => idea: in systems with #concurrent_requests > #cpus, 
%                   => TODO check queueing theory, would be nice to have some actual formulas here ;)
%
%            equalizing by giving a busy stage more CPUs / taking CPUs from a non-busy stage
%            and between the CPUs assigned to a stage, arbitrate by runqueue length

%              => analogy of requests to be handled = water in pipes...?
%              => a CPU will still have threads from the previous stage in the runqueue: doesn't matter because...
%                    ...they will be migrated off this CPU as soon as they schedule out / block on IO
%                    ...round robin ensures the caches are warm for the remaining threads in the old stage
%                    ... we know the requests will finish processing eventually

% Design Deficiencies:
%              if the runqueues are long, this system reacts slowy
%                   => would need to migrate runqueue threads off-cpu 
%                       => could do this lazily in schedule()
%
% Implementation:
%               track c_in per stage in atomic counters
%               ... see patch, fairly boring
%               max_stages, snapshot runqueue counters 
\blindtext

\chapter{Evaluation}\label{ch:eval}
In this section, we present the evaluation of our design.
We present our hardware setup and describe how we verify the supposed effects of our design decisions using various benchmarks.
This section is very closely related to the previous section.
\section{Evaluation Setup}
\section{Fast Thread Migration}
\section{Work Conservation}
\section{Stage-Aware Scheduling Policy}

\chapter{Conclusion}\label{ch:concl}
\blindtext
\section{Future Work}
\begin{itemize}
    \item at ITEC OS Group: automatic profiling \& finding of migration points.
    \item auto-evaluating scheduler: measure if stage migrations actually make sense by computing a break-even point and continously measuring the result of scheduling decisions using performance conuters.
    \item NUMA / SMT-aware Scheduling Policy
    \item MWAIT
\end{itemize}

\backmatter

\chapter{Appendix}
\blindtext
\begin{itemize}
    \item Source code and commit history of our modified version of OSv
    \item Source code and commit history of our modified version of MySQL 5.6
    \item Source code and commit history of our microbenchmarks and measurement scripts
\end{itemize}

\cleardoublepage
\phantomsection
\addcontentsline{toc}{chapter}{Bibliography}
\printbibliography

\end{document}
