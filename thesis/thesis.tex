\documentclass[12pt,a4paper]{article}
\usepackage[utf8]{inputenc}
\usepackage[inline]{enumitem}
\usepackage{parskip} % disable indentation for new paragraphs, increased margin-bottom instead
\usepackage[english]{babel}

\usepackage[style=alphabetic]{biblatex}
\addbibresource{bib.bib}

\usepackage{hyperref}

\usepackage{todonotes}

\title{{\large Bachelor Thesis}\\Stage-aware Scheduling in a Library OS}
\author{Christian Schwarz}
\date{TODO December 2017 -- TODO March 2018}

\begin{document}

\maketitle

\clearpage

\section{Abstract}

\clearpage

\tableofcontents

\clearpage

\section{Introduction}

\section{Related Work}
We explain motivation and related work in the field of cache-aware scheduling policies / application architecture.
This will borrow heaviliy from the expose.

\subsection{Profiling at Google (Kanev et al)}
\subsection{Memory Hierarchy on Contemporary Server-Class CPUs}
\subsection{SEDA}
\subsection{Software Data Spreading}
\subsection{User-space solution at KIT ITEC OS Group}

\section{Analysis}
Description of the implementation and and analysis of the problems observed in the user-space solution on Linux:
\begin{itemize}
    \item Cache miss rate observably reduced
    \item However, net performance is worse
    \item Reason 1: Performance of thread-migration / syscall overhead?
    \item More important resaon2: policy is not work-conserving:
        user-space apporach mandates thread migration must happen in userspace,
        thus threads that become runnable must wait for user space
    \item => fundamental impplementation problem 1: tight scheduler integration is required for a work-consercing solution.
    \item => fundamental implementation problem 2: stage definitions are application specific, the entire scheduling policy is application specific => does not fit into the traditional model of whole-system scheduler like Linux CFS
\end{itemize}

\clearpage

\section{Design \& Implementation}
In this section, we explain the design and implementation of our solution in the OSv library operating system.
The first subsection gives a brief introduction to the concepts of library OSes and the differentiators of OSv in this space.
The remaining subsections present the core ideas of our design and our modifications of OSv to realize them.

\subsection{Overview}
This section gives an overview of our design --- essentially a one-line summary of the contents of the remaining sub-sections plus some pretty diagarams.

\subsection{The OSv Library Operating System}
Short intro to OSv library OS:
\begin{itemize}
    \item Concept of a library OS: vm = app, no user-kernel-boundary => performance promise
    \item Linux ABI-compat
    \item Moderately sized C++ codebase, compact scheduler implementation
\end{itemize}
Idea: OSv allows for easier replacement of the entire scheduling policy and and thread migration mechanisms.
By the nature of a library OS, the changes are kept application-local, leaving fairness considerations up to the hypervisor scheduler (Linux / KVM in our case).

\subsection{Removal of OSv Scheduling Policy}
\subsection{User-Space API}
Mention the stack structure.
\subsection{Fast Thread Migration}
\subsection{Work Conservation}
\subsection{Stage-Aware Scheduling Policy}

\clearpage
\section{Evaluation}
In this section, we present the evaluation of our design.
We present our hardware setup and describe how we verify the supposed effects of our design decisions using various benchmarks.
This section is very closely related to the previous section.
\subsection{Evaluation Setup}
\subsection{Fast Thread Migration}
\subsection{Work Conservation}
\subsection{Stage-Aware Scheduling Policy}

\section{Conclusion}
\subsection{Future Work}
\begin{itemize}
    \item at ITEC OS Group: automatic profiling \& finding of migration points.
    \item auto-evaluating scheduler: measure if stage migrations actually make sense by computing a break-even point and continously measuring the result of scheduling decisions using performance conuters.
    \item NUMA / SMT-aware Scheduling Policy
    \item MWAIT
\end{itemize}

\section{Appendix}

\begin{itemize}
    \item Source code and commit history of our modified version of OSv
    \item Source code and commit history of our modified version of MySQL 5.6
    \item Source code and commit history of our microbenchmarks and measurement scripts
\end{itemize}

\clearpage

\printbibliography

\end{document}
